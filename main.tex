\documentclass[10pt]{beamer}
\usepackage[utf8]{inputenc}
\usepackage{graphicx}
\usepackage{ragged2e}
\usepackage{listings}
\usepackage{minted}
\usepackage{multicol}
\usepackage {mathtools}
\usetheme{CambridgeUS}
\usecolortheme{dolphin}

% set colors
\definecolor{myNewColorA}{RGB}{0, 55, 158}
\definecolor{myNewColorB}{RGB}{0, 55, 158}
\definecolor{myNewColorC}{RGB}{0, 55, 158}
\setbeamercolor*{palette primary}{bg=myNewColorC, fg = white}
\setbeamercolor*{palette secondary}{bg=myNewColorB, fg = white}
\setbeamercolor*{palette tertiary}{bg=myNewColorA, fg = white}
\setbeamercolor*{titlelike}{fg=myNewColorA}
\setbeamercolor*{title}{bg=myNewColorA, fg = white}
\setbeamercolor*{item}{fg=myNewColorA}
\setbeamercolor*{caption name}{fg=myNewColorA}
\usefonttheme{professionalfonts}
\usepackage{natbib}
\usepackage{hyperref}

\newcommand{\specialcell}[2][c]{%
  \begin{tabular}[#1]{@{}c@{}}#2\end{tabular}}
  
\lstdefinelanguage{Kotlin}{
  comment=[l]{//},
  commentstyle={\color{gray}\ttfamily},
  emph={filter, first, firstOrNull, forEach, lazy, map, mapNotNull, println},
  emphstyle={\color{OrangeRed}},
  identifierstyle=\color{black},
  keywords={!in, !is, abstract, actual, annotation, as, as?, break, by, catch, class, companion, const, constructor, continue, crossinline, data, delegate, do, dynamic, else, enum, expect, external, false, field, file, final, finally, for, fun, get, if, import, in, infix, init, inline, inner, interface, internal, is, lateinit, noinline, null, object, open, operator, out, override, package, param, private, property, protected, public, receiveris, reified, return, return@, sealed, set, setparam, super, suspend, tailrec, this, throw, true, try, typealias, typeof, val, var, vararg, when, where, while},
  keywordstyle={\color{NavyBlue}\bfseries},
  morecomment=[s]{/*}{*/},
  morestring=[b]",
  morestring=[s]{"""*}{*"""},
  ndkeywords={@Deprecated, @JvmField, @JvmName, @JvmOverloads, @JvmStatic, @JvmSynthetic, Array, Byte, Double, Float, Int, Integer, Iterable, Long, Runnable, Short, String, Any, Unit, Nothing},
  ndkeywordstyle={\color{BurntOrange}\bfseries},
  sensitive=true,
  stringstyle={\color{ForestGreen}\ttfamily},
}

\setminted[kotlin]{
  linenos=true,
  breaklines=true,
  autogobble,
  encoding=utf8,
  fontsize=\footnotesize,
  frame=lines
}

\beamertemplatenavigationsymbolsempty
%------------------------------------------------------------
%\titlegraphic{\includegraphics[height=1.5cm]{download.png}} 

\setbeamerfont{title}{size=\large}
\setbeamerfont{subtitle}{size=\small}
\setbeamerfont{author}{size=\small}
\setbeamerfont{date}{size=\small}
\setbeamerfont{institute}{size=\small}
\title[]{DevOps per Applicazioni Mobile Multipiattaforma: un Caso di Studio Industriale}

\institute[]{}
\author[Filippo Paganelli]{Alma Mater Studiorum - Università di Bologna \\ Campus di Cesena}
\date[\textcolor{white}{A.A. 21/22}]
{
\begin{columns}[onlytextwidth]
    \begin{column}{0.5\textwidth}
        \begin{flushleft}
            Relatore:\\
            \textbf{Prof. Danilo Pianini}\\
            \vspace{3mm}
            Correlatore:\\
            \textbf{Prof.ssa Catia Prandi}
        \end{flushleft}
    \end{column}
        \begin{column}{0.5\textwidth}
        \begin{flushright}
            Presentata da:\\
            \textbf{Filippo Paganelli}
        \end{flushright}
    \end{column}
\end{columns}
\vspace{10mm}
A.A. 21/22 \\ III Sessione
}

%------------------------------------------------------------
%This block of commands puts the table of contents at the 
%beginning of each section and highlights the current section:
\AtBeginSection[]
{
  \begin{frame}
    \frametitle{Contents}
    \tableofcontents[currentsection]
  \end{frame}
}

%\AtBeginSection[]{
%  \begin{frame}
%  \vfill
%  \centering
%  \begin{beamercolorbox}[sep=8pt,center,shadow=true,rounded=true]{title}
%    \usebeamerfont{title}\insertsectionhead\par%
%  \end{beamercolorbox}
%  \vfill
%  \end{frame}
%}
%------------------------------------------------------------

\begin{document}

%The next statement creates the title page.
\frame{\titlepage}
%\begin{frame}
%\frametitle{Contents}
%\tableofcontents
%\end{frame}
%------------------------------------------------------------

% !TeX root = ../main.tex

\section{Introduzione}

\begin{frame}{Introduzione}
    \begin{columns}[onlytextwidth]
            \begin{column}{0.5\textwidth}
            
            \textbf{Mercato delle applicazioni mobile}:
            \begin{itemize}
                \item Grande crescita
                \item Grande concorrenza
                \item Cambiamenti molto rapidi
                \item Due leader (Android e iOS)
            \end{itemize}
    
            \vspace{3mm}
    
            \textbf{Necessità}:
            \begin{itemize}
                \item Innovazione continua
                \item Ottimizzazione processo
                \item Rilasci frequenti
                \item Alta qualità
            \end{itemize}
            
        \end{column}
        \begin{column}{0.5\textwidth}
        
            \begin{figure}[H]
                \includegraphics[width=1\textwidth]{img/Untitled design.png}
            \end{figure}
            
        \end{column}
    \end{columns}
\end{frame}

\begin{frame}{Problema}

    \textbf{Metodo tradizionale}:
    \begin{itemize}
        \item Stessa applicazione sviluppata due volte
        \item Due team separati
        \item Stesso processo con tecnologie differenti
        \item Comunicazione/collaborazione tra i team
        \item Task manuali
        \item ...
    \end{itemize}

    \vspace{9mm}

    \begin{figure}[H]
        \includegraphics[width=0.7\textwidth]{img/Screenshot 2022-12-05 at 18.01.36.png}
    \end{figure}
    
\end{frame}

\begin{frame}{Soluzione}
    \begin{columns}[onlytextwidth,t]
        \begin{column}{0.45\textwidth}
    
            \textbf{Cultura DevOps}:
            \begin{itemize}
                \item Automazione
                \item Collaborazione
                \item Misurazione
                \item Monitoraggio
            \end{itemize}

            \vspace{4mm}
        
            \textbf{Applicazioni\\Multipiattaforma}:
            \begin{itemize}
                \item Riuso del codice
                \item Risparmio costi/risorse
                \item Manutenzione semplificata
                \item Performance simili al nativo
            \end{itemize}
            
        \end{column}
        \begin{column}{0.55\textwidth}

            \begin{figure}[H]
                \includegraphics[width=1\textwidth]{img/devops-mobile.png}
            \end{figure}
        
        \end{column}
    \end{columns}
\end{frame}
% !TeX root = ../main.tex

\section{Caso di Studio}

\begin{frame}{Contesto Aziendale}
    \begin{figure}[H]
        \includegraphics[width=0.7\textwidth]{img/contesto-aziendale.png}
    \end{figure}
    \begin{itemize}
        \item Core Business: servizi per la pubblica amministrazione e professionisti
        \item Dominio dell'editoria digitale
        \item Necessità di sviluppare un nuovo metodo di accesso alle pubblicazioni digitali che sia più accessibile e pratico
    \end{itemize}
\end{frame}

\begin{frame}{Requisiti: Processo}
    
\end{frame}

\begin{frame}
    \begin{figure}[H]
        \includegraphics[width=0.82\textwidth]{img/full-cicd.png}
    \end{figure}
\end{frame}

\begin{frame}{Requisiti: Applicazione}

\end{frame}
% !TeX root = ../main.tex

\section{Automazione del Processo di Sviluppo}

\begin{frame}{Processo di Sviluppo}

    \begin{figure}[H]
        \includegraphics[width=1\textwidth]{img/sdlc2.png}
    \end{figure}

\end{frame}

\begin{frame}{Pratiche DevOps}
    \begin{columns}[onlytextwidth,t]
        \begin{column}{0.45\textwidth}
    
            \textbf{Continuous Integration}
            \vspace{2mm}
            \begin{itemize}
                \item \textbf{Principio}: Integrazione frequente di piccole modifiche, testate automaticamente
                \vspace{2mm}
                \item \textbf{Stages}:
                \begin{itemize}
                    \item Build
                    \item Test
                    \item Package
                \end{itemize}
            \end{itemize}
            
        \end{column}
        \begin{column}{0.45\textwidth}

            \textbf{Continuous Delivery}:
            \vspace{2mm}
            \begin{itemize}
                \item \textbf{Principio}: Rilascio frequente di piccole modifiche, eseguito in modo automatico
                \vspace{2mm}
                \item \textbf{Stages}:
                \begin{itemize}
                    \item Release alpha
                    \item Release beta
                    \item Release prod
                \end{itemize}
            \end{itemize}
        
        \end{column}
    \end{columns}

    \vspace{2mm}

    \begin{figure}[H]
        \includegraphics[width=0.9\textwidth]{img/cicd.png}
    \end{figure}

\end{frame}

\begin{frame}{Flusso di Lavoro}
    \begin{figure}[H]
        \includegraphics[width=0.8\textwidth]{img/release-flow.png}
    \end{figure}

    \textbf{Caratteristiche}:
    \begin{itemize}
        \item 3 branch principali (dev, test, main)
        \item 3 ``ambienti'' associati (alpha, beta, prod)
        \item Riutilizzabile (Template)
        \item Estendibile (Pipeline-as-Code)
    \end{itemize}
\end{frame}

\begin{frame}{Sistema Complessivo}

    \begin{figure}[H]
        \includegraphics[width=0.85\textwidth]{img/full-cicd.png}
    \end{figure}

\end{frame}
% !TeX root = ../main.tex

\section{Sviluppo Applicazione}

\begin{frame}{Processo di Sviluppo}
    \begin{columns}[onlytextwidth]
        \begin{column}{0.4\textwidth}
            \textbf{Progettazione}:
            \begin{itemize}
                \item Modellazione dominio (editoria digitale)
                \item Progettazione UX/UI
                \item Definizione architettura
            \end{itemize}
            \vspace{3mm}
            \textbf{Implementazione}:
            \begin{itemize}
                \item Ricerca Librerie
                \item Sviluppo logica applicativa (modulo condiviso)
                \item Sviluppo UI Android
                \item Sviluppo UI iOS
            \end{itemize}
        \end{column}
        \begin{column}{0.6\textwidth}
             \begin{figure}[H]
                \includegraphics[width=1\textwidth]{img/stack_kmm.png}
            \end{figure}
        \end{column}
    \end{columns}
\end{frame}

\begin{frame}{Logica Applicativa}
    
\end{frame}

\begin{frame}{UI Android}
    
\end{frame}

\begin{frame}{UI iOS}
    
\end{frame}
% !TeX root = ../main.tex

\section{Conclusioni}

\begin{frame}{Conclusioni}

    \textbf{Risultati raggiunti}:
    \begin{itemize}
        \item Breve time-to-market
        \item 3 mesi da inizio sviluppo a distribuzione applicazione, sia Android che iOS
        \item 28 minuti (media) dalla modifica del codice al rilascio, sia Android che iOS
        \item Riuso/Estensione del processo di sviluppo e del sistema d'automazione
    \end{itemize}

    \vspace{3mm}

    \textbf{Lavori futuri}:
    \begin{itemize}
        \item Realizzazione stesso caso di studio con metodologia crosspiattaforma
        \item Confronto multipiattaforma vs crosspiattaforma
    \end{itemize}
    
\end{frame}


\begin{frame}
    \begin{center}
        \LARGE Grazie dell'attenzione
    \end{center}
\end{frame}

\end{document}